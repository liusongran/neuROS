
%\chapter{}


\chapter[致谢]{致\quad 谢}\chaptermark{致\quad 谢}% syntax: \chapter[目录]{标题}\chaptermark{页眉}
% \thispagestyle{noheaderstyle}% 如果需要移除当前页的页眉
%\pagestyle{noheaderstyle}% 如果需要移除整章的页眉

\reviewORprint{
}{

六年的博士生活即将结束,在这里我由衷地感谢多年来给予我无私帮助和关怀的老师、同学、朋友和家人。

首先要感谢我的导师王义教授。是您让我有机会来到智慧系统实验室这个大集体,是您引领我进入了科研的大门。感谢您在我的学习生涯中提供的宝贵支持、指导和建议。王老师严谨的治学态度让我受用一生,做学问一定要踏实认真,精益求精,不要一味追求出成果,要懂得过程的重要性。同时,您工作和生活中充沛的精力同样感染着我,教会了我永不放弃,保持一颗积极乐观的态度去面对多彩的学习和生活。您的治学风格和人格魅力使我受益终生。在此向我的导师王义教授表示衷心的感谢与祝福!

感谢关楠教授。在您身边受教已近五载,您不但提供给我高水平的研究平台,还对我的工作进行了大量具体的指导。您的诸多教诲,诸如“换个角度看问题”,“只要肯思考,办法总比困难多”,“不要逃避问题”等等让我受益匪浅。在未来的工作中,我会以您为榜样,谨记您的教诲,知行合一,自强不息!

感谢吕鸣松教授。您不但在工作和生活上给予我大量的帮助,更在为人处世上传授我诸多经验,使我的性格更加成熟。您认真严谨的学术态度是我学习的榜样,在此向吕老师表达我最诚挚的感谢!

感谢其他各位老师、同学和朋友的关心与帮助。能够生活和学习在这样温馨的大集体中,是我人生难忘的经历。感谢魏阳杰教授在我攻读博士期间对我的关心和照顾,您的谦逊和乐观是我学习的榜样。感谢东北大学智慧系统实验室的同学王样,唐月,纪东等等,能够和你们一起成长使我倍感幸运。感谢访问香港理工大学期间的同学姜徐,何青强等,你们的聪明才智给我的研究工作带来了取之不尽的灵感。特别感谢张伟同学,在我学术遇到困难时给予的支撑与帮助。特别感谢刘卓然同学,相识十载我们一起度过了愉快的大学及博士时光,感谢你在我学习生活遇到困难时给予的支持和鼓励。感谢刘松冉同学,多年来在工作、学习和生活方面对我的帮助,我很幸运能遇到你这样的好师兄。还有诸多好友,篇幅所限无法一一列举诸位名姓,感谢你们的每一次帮助!

感谢我的父母。你们无私的爱与包容,是我坚强的后盾和前行的动力。你们的恩情我穷尽一生也无法报答,我定当竭力进取以求不负你们之厚望。并祝愿我的亲人们身体健康,喜乐长安。

还要感谢我的爱人陶力辉。在我学术生活遇到困难时,给予理解与安慰。感谢你的体谅与包容,这本薄薄的论文是献给你的礼物。

最后要感谢评阅本论文的各位老师及朋友,谢谢你们抽出了宝贵的时间并给予我中肯的建议。

}

\chapter{攻读博士学位期间取得的学术成果}


\reviewORprint{
    \section*{第一作者发表/录用学术论文:}
    \begin{enumerate}[leftmargin=*]
        \item Title[J]Journal, Year, Volume(Number):Pages. (\textbf{SCI, JCR1, 论文第三章})
    \end{enumerate}
    \section*{第一作者在审学术论文:}
    \begin{enumerate}[leftmargin=*]
        \item Title[J]Journal, Year, Volume(Number):Pages. (\textbf{SCI, JCR1, 论文第三章})
    \end{enumerate}
    \section*{通讯作者发表/录用学术论文:}
    \begin{enumerate}[leftmargin=*]
        \item Title[J]Journal, Year, Volume(Number):Pages. (\textbf{SCI, JCR1, 第二作者/通讯作者})
    \end{enumerate}
    \section*{合作作者发表/录用学术论文:}
    \begin{enumerate}[leftmargin=*]
        \item Title[J]Journal, Year, Volume(Number):Pages. (\textbf{SCI, JCR1, 第二作者})
    \end{enumerate}
}{
    \section*{学术论文:}
    \begin{enumerate}[leftmargin=*]
        \item \textbf{He Du}, Wei Zhang, Nan Guan, Wang Yi. Scope-aware data cache analysis for OpenMP programs on multi-core processors[J]. Journal of Systems Architecture, 2019, Volume(98): 443--452. (SCI检索, CCF-B, 论文第四章)
        \item \textbf{He Du}, Xu Jiang, Tao Yang, Mingsong Lv, Wang Yi. Real-Time Scheduling and Analysis of OpenMP Programs with Spin Locks[C]. IEEE International Conference on Parallel and Distributed Systems (ICPADS), Hong Kong, 2-4 December, 2020. (EI, CCF-C, 论文第五章)
        \item \textbf{He Du}, Xu Jiang, Mingsong Lv, Tao Yang, Wang Yi. Scheduling and Analysis of Real-Time Task Graph Models with Nested Locks[J]. Journal of Systems Architecture, Volume 114, March 2021, 101969. (SCI, CCF-B, 论文第二章)
        \item Xu Jiang, Nan Guan, \textbf{He Du},  Weichen Liu, Wang Yi. On the Analysis of Parallel Real-Time Tasks with Spin Locks[J]. 2020 IEEE Transactions on Computers, Volume(67): 975--991. (SCI, CCF-A, 论文第三章)

    \end{enumerate}
}

\section*{科研项目:}

\begin{enumerate}[leftmargin=*]
    \item 国家自然科学基金重点项目:混合关键型多核嵌入式软件设计、验证与优化关键技术研究(67532007),2016年1月至2020年12月,主要参与者。
    \item 国家自然科学基金重点项目:面向GPU的实时系统时间分析与优化技术研究(61772123),2018年1月至2021年12月,主要参与者。
\end{enumerate}

\forget{

\chapter{个人简历}
杜贺,女,汉族,1993年6月出生于辽宁省沈阳市。2011年考入东北大学信息科学与工程学院计算机专业,于2015年7月毕业,获得工学学士学位。同年保送本校博士研究生,师从王义教授和关楠教授。主要从事实时系统中实时调度算法及并行程序分析的研究工作。2017年至今访问香港理工大学,师从关楠教授。

2015年至今,致力于系统中实时调度算法设计、共享资源的实时锁协议设计与并行程序分析等研究。攻读博士学位期间,共发表论文4篇,其中第一作者被SCI收录2篇,第一作者被EI收录1篇。研究成果发表在ICPADS、JSA等国际著名会议和期刊中。
%作为项目主要参与人己完成和正在参与的科研项目主要有国家自然科学基金:混合关键系统动态实时调度与容错设计研究(67532007),面向GPU的实时系统时间分析与优化技术研究(61772123)等。

}

\cleardoublepage[plain]% 让文档总是结束于偶数页,可根据需要设定页眉页脚样式,如 [noheaderstyle]

